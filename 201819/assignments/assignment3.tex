\documentclass[a4paper,12pts]{article}

\usepackage[english]{babel}
\usepackage[latin1]{inputenc}
%\usepackage[dvips]{graphicx}
\usepackage{graphicx}
\usepackage{amsmath}
\usepackage{amssymb}
\usepackage{amsthm}
\usepackage{color}
\usepackage{hyperref}


\begin{document}

\title{Assignment 3: Content Based Image Retrieval \\
Vision and Image Processing}
\author{S�ren Olsen and  Fran\c{c}ois Lauze}
\date{December 10, 2018}
\maketitle


\noindent 
This is the third mandatory assignment on the course Vision
and Image Processing. The goal for you is to get familiar with
the Bag Of Words (BoW) principle and its use in content based image
retrieval systems.
\bigskip

{\bf This assignment must be solved in groups}. We expect that you will
form small groups of 2 to 4 students that will work on this assignment.
You have to pass this and the other 3 mandatory assignments in
order to pass the course.   If you do not pass this assignment, but you
have made a SERIOUS attempt, you will get a second chance of
submitting a new solution. 
\bigskip

{\bf The deadline for this assignment is Monday 10/1, 2019 at 23:59}. 
You must submit your solution electronically via the Absalon home
page. For general information on relevant software, requirement to the form
of your solution including the maximal page limit, how to upload on
Absalon etc, please see the first assignment.


\section*{Bag of Visual Words Content Based Image Retrieval}
The goal of the assignment is to implement a prototypical CBIR
system. We recommend the use of the
\href{http://www.vision.caltech.edu/Image_Datasets/Caltech101}{CalTech
  101} image database
\url{http://www.vision.caltech.edu/Image_Datasets/Caltech101/}.
% One alternative is
% \href{http://www.vis.uky.edu/~stewe/ukbench}{University of Kentucky
%  Benchmark Data Set} but it is huge and has only four images per
% categories.  
We recommend that you (for a start) select a subset of say 4-5 categories.  When you have
checked that everything works you may extend to say 20 categories. \textit{Beware that
  with many categories and training images, the codebook generation via clustering might
be very computationally intensive}.  For each category, the set of images should be split in
two: A training set and a test set (of equal size). The test set must not include images
in the training set. When using few categories you may also limit the number of training
images (to say 10) per category. Depending on your amount of computational power, for more
categories, you may increase the number of training images to the double or more.
\medskip

You should extract visual words using SIFT descriptors 
(ignoring position, orientation and scale) or similar descriptors
extracted at interest points. To compute the descriptors, we recommend
to use VLFeat's \texttt{sift}, but other options are possible.


\section{Codebook Generation}
In order to generate a code book, select a set of training images. Then Extract SIFT
features from the training images (ignore position, orientation and scale). The SIFT
features should be concatenated into a matrix, one descriptor per row, i.e., if you use
$n$ training images, you will concatenate the descriptors into one single matrix.

Then you should run the $k$-means clustering algorithm on the subset of training
descriptors to extract good prototype (visual word) clusters.  A reasonable $k$ should be
small (say between 200 and 500) for a small number of categories (say 5) and larger (say
between 500 and 2000) for a larger number of categories. Also, a good value of $k$ may
depend on the complexity of your data.  You should experiment with a few different values
of $k$ (but beware that this can be rather time-consuming).  \medskip

Once clustering has been obtained, classify each training descriptor to
the closest cluster centers) and form the bag of words (BoW) for each
image in the image training set.  
\medskip

For extraction of the SIFT features in python, using Python3 and OpenCV,
here a simple example from the WWW\footnote{https://www.pyimagesearch.com/2015/07/16/where-did-sift-and-surf-go-in-opencv-3/}:
\begin{verbatim}
$ python
>>> import cv2
>>> image = cv2.imread("test_image.jpg")
>>> gray = cv2.cvtColor(image, cv2.COLOR_BGR2GRAY)
>>> sift = cv2.xfeatures2d.SIFT_create()
>>> (kps, descs) = sift.detectAndCompute(gray, None)
>>> print("# kps: {}, descriptors: {}".format(len(kps), descs.shape))
# kps: 274, descriptors: (274, 128)
\end{verbatim}


For the $k$-means algorithm, you should use one of the implementations provided by OpenCV, \texttt{scipy.clusters.vq.kmeans} 
or Scikit-Learn implementation: \texttt{sklearn.cluster.k\_means}.

\section{Indexing}
The next step consists in content indexing. For each image in the
test set you should:

\begin{itemize}
\item Extract the SIFT descriptors of the feature points in the image,
\item Project the descriptors onto the codebook, i.e., for each
  descriptor the closest cluster prototype should be found,
\item Construct the generated corresponding bag of words, i.e, word histogram.
\end{itemize}

Please note that you have already performed the same steps for the
training images during codebook generation.
\medskip

The result should be saved in a table that would contain, per entry at least
the file name, the true category, if it belongs to the training- or
test set,  and the corresponding bag of words / word histogram.
The table need only be computed once and then used repeatably in the
following retrieval experiments. 


\section {Retrieving}
Finally, you should implement retrieving of images using some of the
similarity measures discussed in the course slides. You may use:
\begin{itemize}
\item common words
\item tf-ifd similarity
\item Bhattacharyya distance or Kullback-Leibler divergence
\end{itemize}

Your report should show commented results for two experiments.
In the first you consider retrieving training images. In the second
you test how well you can classify test images.  Otherwise the two
test are identical.  For each test you should count:
\begin{itemize}
\item The mean reciprocal rank (i.e. the average across all queries of
   $1/\mbox{rank}_i$, where $\mbox{rank}_i$ is the rank position of the
   first correct category for the i'th query).
\item How often (in per cent) the correct category is in top-3
\end{itemize}

Please note that the measures above are just two among a long list of
possible performance measures.  If you google {\em Information
  retrieval} you may find alternative measures.

\end{document}
