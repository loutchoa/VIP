\documentclass[a4paper,12pts]{article}

\usepackage[english]{babel}
\usepackage[latin1]{inputenc}
%\usepackage[dvips]{graphicx}
\usepackage{graphicx}
\usepackage{amsmath}
\usepackage{amssymb}
\usepackage{amsthm}
\usepackage{color}
\usepackage{url}
\usepackage{hyperref}


\begin{document}

\title{Assignment 1: Filtering and edge detection\\
Vision and Image Processing}
\author{S�ren Olsen, Francois Lauze}
\date{\today}
\maketitle


\noindent 
This is the first mandatory assignment on the course Vision and Image
Processing. The goal for you is to get familiar  with computer vision
programming with a focus on filtering. Filtering is a core 
discipline in image processing and forms the basis for feature
extraction which again is central to many of the techniques we are
going to study.
\bigskip

{\bf This assignment must be solved individually}.  You have to pass
this and the following mandatory assignments and quizzes in order to
pass this course.  If you do not pass the assignment, but you have
made a SERIOUS attempt, you will get a second chance of submitting a
new solution.  
\bigskip

{\bf The deadline for this assignment is Wednesday 6/12, 2017 at 22:00}. 
You must submit your solution electronically via the Absalon homepage. Go to
the assignments list and choose this assignment and upload your
solution prior to the deadline.  Remember to include your name and KU
user name (login for {KU}net) in the solution. 

\subsection*{Filtering and edge detection}
Filtering (really Finite Input Responce filtering) is done by
convolutions. Among the most applied filters are the Gaussian and its
first and second order derivatives. Convolution with a Gaussian itself
will blurr the image.  Convolutions with the two first order Gaussian
derivatives provides an estimate of the gradient field from which the
gradient magnitude may be derived and visualized. Convolution with the
sum of the second order (unmixed) partial derivatives of a Gaussian
(dubbed The mexican hat operator) may be used both to detect blobs (as
the local extremes) and edges (as the zero crossings). Edges and
blobs in images code much of the semantic information available in
the images and are often used as building elements in further analysis.
\medskip

In this assignment you must demonstrate that you can write programs
for image filtering and edge detection. Please check the note below on
the use of relevant software.  In detail you must demonstrate that you
and implement, perform and evaluate: 
\newpage

\begin{itemize} 
\item Gaussian filtering. Show the result using $\sigma = 1, 2, 4, 8$
  and explain in detail what can be seen.
\item Gradient magnitude computation using Gaussian derivatives.
  Use $\sigma = 1, 2, 4, 8$, and explain in detail what can be seen
  and how the results differ.
\item Laplacian-Gaussian- (= Mexican hat-) filtering. Again,  use 
$\sigma = 1, 2, 4, 8$, and explain in detail what can be seen and how
  the results differ. 
\item Canny (or similar) edge detection. Select what you think is a
  set of good parameter values, apply, show, and compare with your
  previous results. 
\end{itemize}

You should apply the abovementioned methods to the image 
{\bf lenna.jpg} available at Absalon. However, you are encouraged to
use other images as well (in particular very simple images, that more
easily let you verify your result).  Please notice that for display
purpose some filtering results probably need to be remapped into [0:1]
or whatever your display routine requires. Please avoid using strange
color coding.
\bigskip

Unless explicit stated otherwise, during this course you are allowed
to apply routines available in any library that you may find useful.
For this assignment in particular, you are not supposed to implement
neither Canny edge detection nor Gaussian convolutions (although the
latter might be a good programming exercise). Please see the note on
relevant software below. Also, please recognize that the less you
demonstrate your ability to write relevant (non-trivial) code, the
more you are expected to conduct experiments and (in particular) to
comment on these and to present your new knowledge gained by the
experiments. 
\medskip

As described in detail below your answer should include your code
and a pdf-file describing your solution and showing examples of
well-chosen image results.  Each image/graph or other illustration 
should be commented (don't trust that I see what you see).  In detail
a  solution must include:

\begin{enumerate}
\item  A PDF file describing your answers to the assignment,
  which may include images, graphs and tables if needed. You are
  allowed to used {\bf Max 8 pages} of text including figures and
  tables). We recommend to write only 2-3 pages of text excluding all
  graphics.  Images should be shown large enough to show what you want
  me to see (if printet on A4 paper). Do NOT include your source code
  in this PDF file. 

\item A  zip-file containing: 
\begin{itemize}
\item Your solution source code in original plain text format (Matlab 
  / Python scripts / C / C++ code) with comments about the major steps
  involved in each question.  
\item Your code should be structured such that there is one main file
  that we can run to reproduce all the results presented in your
  report. This main file can, if you like, call other files with
  functions / classes.
\item A README text file describing how
  to compile (if relevant) and run your program, as well as a list of
  all relevant libraries needed for compiling or using your code. If
  we cannot make your code run we will consider your submission
  incomplete and you may be asked to resubmit.
\end{itemize}
\end{enumerate}

Please notice that all documents (except your main answer to the
assignment) including all code must be put into a single compressed
archive file in the ZIP format (RAR is not allowed - we simply cannot
read your archive). 

 

% Extra: Consider extending your solution to multi-scale detectors such
% as the multi-scale Laplacian, multi-scale Harris corner or
% Harris-Laplacian detectors.


\subsection*{A note on relevant software}
We recommend that you select python 2.7 as your programming
language. However, Matlab, C,  C++, or Python 3.5 or Python 3.6 may be
other possibilities. We also recommend that you select the language you are 
most confident in. The focus should be on learning the methods and not 
learning a new programming language.
\medskip

We strongly recommend that you install and use the OpenCV library
\url{http://opencv.org/} and/or the VLFeat library
\url{http://www.vlfeat.org/}. Both libraries provide an extensive
collection of implementations of central computer vision
algorithms.  OpenCV provides an interface for Python 2.7.  It seems
that OpenCV also might be installed under Python 3.5 and 3.6. However, the
installation process might be less straight forward. For Mac OSX we
recommend using MacPorts for Python 2.7, OpenCV and other libraries. 
\medskip

If you wish to use Matlab, you may download and install this from the
``Softwarebiblioteket'' available at KUnet.


\end{document}