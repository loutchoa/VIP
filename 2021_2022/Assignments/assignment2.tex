\documentclass[a4paper,12pts]{article}

\usepackage[english]{babel}
\usepackage[latin1]{inputenc}
%\usepackage[dvips]{graphicx}
\usepackage{graphicx}
\usepackage{amsmath}
\usepackage{amssymb}
\usepackage{amsthm}
\usepackage{color}
\usepackage{url}
\usepackage{hyperref}


\begin{document}

\title{Assignment 2: Filtering and edge detection\\
Vision and Image Processing}
\author{Søren Olsen, Francois Lauze}
\date{\today}
\maketitle


\noindent 
This is the second mandatory assignment on the course Vision and Image
Processing. The goal for you is to get familiar  with computer vision
programming with a focus on filtering. Filtering is a core 
discipline in image processing and forms the basis for feature
extraction which again is central to many of the techniques we are
going to study.
\bigskip

% {\bf This assignment must be solved individually}.
You have to pass
this and the following mandatory assignments % and quizzes
in order to
pass this course.  If you do not pass the assignment, but you have
made a SERIOUS attempt, you will get a second chance of submitting a
new solution.  
\bigskip

{\bf The deadline for this assignment is Wednesday 8/12, 2021 at 22:00}. 
You must submit your solution electronically via the Absalon homepage. Go to
the assignments list and choose this assignment and upload your
solution prior to the deadline.  Remember to include the names and KU
user names (login for {KU}net) for all group members in the solution.
Please find a decription at Absalon of how you register your group for
upload of your assignment answer.


\subsection*{Filtering and edge detection}
Filtering (really Finite Input Responce filtering) is done by
convolutions. Among the most applied filters are the Gaussian and its
first and second order derivatives. Convolution with a Gaussian itself
will blurr the image.  Convolutions with the two first order Gaussian
derivatives provides an estimate of the gradient field from which the
gradient magnitude may be derived and visualised. Convolution with the
sum of the second order (unmixed) partial derivatives of a Gaussian
may be used both to detect blobs (as
the local extremes) and edges (as the zero crossings). Edges and
blobs in images code much of the semantic information available in
the images and are often used as building elements in further analysis.
\medskip

In this assignment you must demonstrate that you can write programs
for image filtering and edge detection. Please check the note below on
the use of relevant software.  In detail you must demonstrate that you
and implement, perform and evaluate: 
\newpage

\begin{itemize} 
\item Gaussian filtering. Show the result using $\sigma = 1, 2, 4, 8$
  pixels and explain in detail what can be seen.
\item Gradient magnitude computation using Gaussian derivatives.
  Use $\sigma = 1, 2, 4, 8$ pixels, and explain in detail what can be
  seen and how the results differ.
\item Laplacian-Gaussian filtering. You may implement this as a
  difference og Gaussians. Again,  use $\sigma = 1, 2, 4, 8$ pixels,
  and explain in detail what can be seen and how the results differ. 
\item Canny (or similar) edge detection. Describe the parameter values
  and their impact on the result. Select what you think is a set of
  good parameter values, apply, show and decribe the result.
\end{itemize}

Gaussian convolutions may be implemented in the spatial/pixel-domain
or in the Fourier frequency domain.  You may find an implementation of
the latter at Absalon. In OpenCV routines for spatial Gaussian
convolutions are available. For such implementations the filter size
must be provided. Please note that independent of the choice of
$\sigma$, the Gaussian is never zero. This means that even values at
the extreme ends of the bell curve have an impact on the result of the
convolution. However, for practical applications, the filter size may
be truncated to a finite size at about $\pm 3.0 \sigma$ along each
dimension. Also note that $\sigma$-values below about 0.5 do not make
sense when applied to the discrete pixel grid of an image. Similarly, the
first and second derivatives of a Gaussian may be truncated at about
$\pm 3.4 \sigma$ and $\pm 3.8\sigma$. We recommended that you verify
these values visually, by plotting 1D versions of the Gaussian and its
derivatives.  \bigskip 

You should apply the above-mentioned methods to the image 
{\bf lenna.jpg} available at Absalon. However, you are encouraged to
use other images as well (in particular very simple images [say a
black square on a white background), that more easily let you verify
your result).  Please notice that for display purpose some filtering
results probably need to be remapped into [0:1] or whatever your
display routine requires. Please avoid using strange color coding.
\bigskip

Unless explicit stated otherwise, during this course you are allowed
to apply routines available in any library that you may find useful.
For this assignment in particular, you are not supposed to implement
neither Canny edge detection nor Gaussian convolutions (although the
latter might be a good programming exercise). Please see the note on
relevant software below. Also, please recognise that the less you
demonstrate your ability to write relevant (non-trivial) code, the
more you are expected to conduct experiments and (in particular) to
comment on these and to present your new knowledge gained by the
experiments. 
\medskip

As described in detail below your answer should include a pdf-file
describing your solution and showing examples of well-chosen image
results. In addition you should include your code as a zip-file.  Each
image/graph or other illustration should be commented (don't trust
that I see what you see).  In detail a  solution must include:

\begin{enumerate}
\item  A PDF file describing your answers to the assignment,
  which may include images, graphs and tables if needed. You are
  allowed to used {\bf Max 5 pages} of text including figures and
  tables).  Images should be shown large enough to show what you want
  me to see. Do NOT include your source code  in this PDF file. 

\item A  zip-file (uploaded as the second file) containing: 
\begin{itemize}
\item Your solution source code in original plain text format (Matlab 
  / Python scripts / C / C++ code) with comments about the major steps
  involved in each question.  
\item Your code should be structured such that there is one main file
  that we can run to reproduce all the results presented in your
  report. This main file can, if you like, call other files with
  functions / classes.
\item A README text file describing how
  to compile (if relevant) and run your program, as well as a list of
  all relevant libraries needed for compiling or using your code. If
  we cannot make your code run we will consider your submission
  incomplete and you may be asked to resubmit.
\end{itemize}
\end{enumerate}

Please notice that all documents (except your main answer to the
assignment) including all code must be put into a single compressed
archive file in the ZIP format (RAR is not allowed - we simply cannot
read your archive). 

 

\subsection*{A note on relevant software}
We recommend that you select the most recent version of Python
including relevant libraries such as OpenCV. We recommend that you
select the language you are most confident in. The focus should be on
learning the methods and not learning a new programming language. When
using software from any library please state which routiones were
applied when and from which library. Also specify parameter settings etc.

\end{document}