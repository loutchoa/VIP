\documentclass[a4paper,12pts]{article}



\usepackage{francois-preamble}
\usepackage{hyperref}
\usepackage[english]{babel}
\usepackage[latin1]{inputenc}



\begin{document}

\title{Assignment 5: Photometric Stereo \\
Vision and Image Processing}
\author{ Fran\c{c}ois Lauze and S�ren Olsen}
\date{January 3, 2018}
\maketitle


\noindent 
This is the fourth mandatory assignment on the course Vision
and Image Processing. The goal is to implement some basic Photometric Stereo.
\bigskip

{\bf This assignment must be solved in groups}. We expect that you will
form small groups of 2 to 4 students that will work on this assignment.
You have to pass this and the other 3 mandatory assignments in
order to pass the course. 
\bigskip

{\bf The deadline for this assignment is Wednesday 17/1, 2018 at 20:00}. 
You must submit your solution electronically via the Absalon home
page. For general information on relevant software, requirement to the form
of your solution including the maximal page limit, how to upload on
Absalon etc, please see the first assignment.

\section*{Photometric Stereo}

The goal of this assignment is to implement basic photometric
stereo. For that you will use two datasets provided to you on Absalon in the \texttt{Assignmen 5 Code and data} folder,
\texttt{Beethoven.mat} and \texttt{Buddha.mat}. The first one consists of 3 images, and is a synthetic example,
while the second one consists of 10 images and is a real one.
\medskip\\
\emph{Note on datasets and softwares.\/} Datasets are stored in Matlab
mat-files. Each file contains the following variables:
\begin{itemize}
\item a 3D array \texttt{I} of size $(m,n,k)$ where $(m,n)$ is the
  size of each image and $k$ is the number of views, i.e., view $i$
  corresponding to lighting $\bs_i$ is \texttt{I(:,:,i)}.
\item a 2D binary array \texttt{mask} of size $(m,n)$. Pixels with
  values 1 (\texttt{true}) indicate positions where intensity data has
  been recorded. Photometric Stereo should only be solved at these
  points.
\item an array $\texttt{S}$ of light vectors, of size $(k,3)$, where
  line $i$ represents the directional light $\bs_i$ that was used to
  obtain image \texttt{I(:,:,i)}.
\end{itemize}

Software for integration of the normal field and surface display is provided both for
Matlab and Python. For Matlab, 3 functions are provided:
\begin{itemize}
\item \texttt{function z = unbiased\_integrate(n1, n2, n3, mask)}
  computes a depth map for the normal field given by $(n1,n2,n3)^T$
  only within the mask using a so-called ``Direct Poisson Solver''.
  The resulting array \texttt{z} has the same size as
  \texttt{mask}. Values that correspond to pixel locations where
  \texttt{mask == 0} are set to \texttt{nan} (Not a Number).
\item \texttt{function z = simchony\_integrate(n1, n2, n3, mask)}
  computes a depth map for the normal field given by $(n1,n2,n3)^T$
  a Fourier-Transform based solver. As for \texttt{unbiased\_integrate}, 
  the resulting array \texttt{z} has the same size as
  \texttt{mask}. Values that correspond to pixel locations where
  \texttt{mask == 0} are set to \texttt{nan} (Not a Number).
\item \texttt{function display\_depth(z)} displays the obtained depth
  map \texttt{z} as a 3D graph.
\end{itemize}

For Python, a module called \texttt{ps\_utils.py} is provided. It
contains functions similar to the Matlab ones and an extra one.
\begin{itemize}
\item \texttt{unbiased\_integrate(n1, n2, n3, mask)}, works as the Matlab one,
\item \texttt{simchony\_integrate(n1, n2, n3, mask)}, works as the Matlab one.
\item \texttt{display\_depth(z)}, works more or less as Matlab
  one. \textsc{Beware}: it requires Python's module/package
  \texttt{mayavi}.
\item \texttt{read\_data\_file(filename)} reads a dataset Matlab
  mat-file and returns \texttt{I}, \texttt{mask} and \texttt{S}.
\item It also contains a few extra functions that should not
  necessarily be of interest to you.
\end{itemize}
\medskip
For both \texttt{Beethoven} and \texttt{Buddha} datasets, data
manipulation and reshaping is very important to maintain good
performances.  Both Matlab and Python/Numpy allow you to extract a
subarray as a list of value, and affects a list of value to an image
subarray. You may want to look at integration source code to see how
it can be done.



\section{Beethoven Dataset}

\texttt{Beethoven} is a synthetic and clean dataset, with exactly 3
images. If $nz$ is the number of pixels inside the non-zero part of
the mask, You should create an array $J$ of size/shape $(3,nz)$ and
obtain the albedo modulated normal field as $M=S^{-1}J$. With it,
extract the albedo within the mask, display it as a 2D image.  Then
extract the normal field by normalizing $M$, extract its components
$n1$, $n2$, $n3$. Solve for depth and display it at different view
points.



\section{Buddha Dataset}

\texttt{Buddha} is real dataset, with exactly 10
images. If $nz$ is the number of pixels inside the non-zero part of
the mask, You should create an array $J$ of size/shape $(10,nz)$ and
obtain the albedo modulated normal field as $M=S^\dagger J$ (the pseudo-inverse). With it,
extract the albedo within the mask, display it as a 2D image.  Then
extract the normal field by normalizing $M$, extract its components
$n1$, $n2$, $n3$. Solve for depth and display it art different view
points.

The result might be disappointing! Suggest other possibilities for
handling this dataset (as mentioned in the lectures). Feel free to implement them:-)




\end{document}
