\documentclass[a4paper,12pts]{article}

\usepackage[english]{babel}
\usepackage[latin1]{inputenc}
%\usepackage[dvips]{graphicx}
\usepackage{graphicx}
\usepackage{amsmath}
\usepackage{amssymb}
\usepackage{amsthm}
\usepackage{color}
\usepackage{hyperref}


\begin{document}

\title{Assignment 3: Stereo correspondence analysis \\
Vision and Image Processing}
\author{S�ren Olsen, Francois Lauze}
\date{\today}
\maketitle

\begin{center}
\includegraphics[width=5.6cm]{Images/imL.png} 
\hspace{2mm}
\includegraphics[width=5.6cm]{Images/imR.png} 
\end{center}


\noindent 
This is the third mandatory assignment on the course Vision
and Image Processing. The goal for you is to get familiar with
dense stereo matching for disparity computation.
\bigskip

{\bf This assignment must be solved in groups}. We expect that you will
form small groups of 2 to 4 students that will work on this assignment.
You have to pass this and the other 3 mandatory assignments in
order to pass the course.   If you do not pass this assignment, but you
have made a SERIOUS attempt, you will get a second chance of
submitting a new solution. 
\bigskip

{\bf The deadline for this assignment is Monday 8/1, 2018 at 22:00}. 
You must submit your solution electronically via the Absalon home
page. For general information on relevant software, requirement to the form
of your solution including the maximal page limit, how to upload on
Absalon etc, please see the first assignment.


\section*{Stereo}
The goal of this assignment is to implement and test a traditional
algorithm for dense stereo correspondence analysis.  You may assume
that the stereo image pairs show richly textured piecewise smooth
surfaces with Lambertian reflectance properties and that the images are
rectified, i.e. corresponding point have the same vertical coordinate
value. In this assignment you are not supposed to compute depth, but
only disparities, i.e. the difference in horizontal coordinate. 
\medskip

Some images (see later) may be in color and some in gray. You are
welcome to base your matching procedure on color information or to
attribute each pixels with a longer vector of feature values
(e.g. obtained by local filtering). You may also stick to intensity
matching. Color conversion from RGB to luminance Y or intensity I may
be done using  $Y \;=\; 0.3*R + 0.6*G + 0.1*B$, and $I = (R+G+B)/3$. 
\medskip

You may find that the descriptive strength of single pixel intensity
information is limited (and may not always single out the correct
matches). To increase the descriptive power you should use a local
window of pixel values. More concretely, you should compute the data
term (measuring the degree of intensity agreement) by normalised cross
correlation. You should vary the size of the window, say from 5x5 over
7x7 to 11x11, and report what you think is the optimal size.  One
approach to reduce the number of false matches is to apply two-way
matching and only to accept candidate matches that agree.  Another
approach is to require that the best match candidate is much better
than the second best candidate. We recommend that you do {\bf not}
experiment with such techniques before you have you have a more simple
functioning program.
\medskip

You will probably need to apply the knowledge that the scene surface
is piecewise smooth. Thus, the candidate match with smallest data term
is not necessarily the correct/best if it result in a disparity value
deviating a lot from the local mean/median disparity value. One method
is to replace all such deviating disparity estimates with a local
mean/median for disparity. Alternatively, you may somehow weight
the data term and the smoothness term together into one measure of
goodness and select the best match.
\medskip

To handle numerically large disparities a coarse-to-fine approach
using  a (truncated) factor-2 pyramid representation of the images
should be applied. Here, each image in the pyramid is subsampled with
a factor of 2 (along each dimension) from the next larger image. The
disparity magnitude will be reduced with the same factor. Thus, the
disparity will be numerically small at the topmost pyramid level and 
a small search area (of $\pm M$) will suffice.
\medskip

Having estimated the disparity at the top pyramid level you may
upsample the reconstructed disparity image to the size of the images
at the pyramid level just below and double the disparity values. This
gives you a prior for the disparity at this pyramid level and makes
possible to keep the search constantly small (e.g. $\pm M$).  
\medskip

Repeating the above procedure iteratively down through the pyramid
levels you may finally obtain matches at the ground pyramid level and
reconstruct a dense disparity surface at the original image
resolution. In this assignment you are  advised to keep the number of
pyramid levels fixed to say 4. This will make it possible to register
disparity values as large as $D_{\mbox{max}} \;=\; M \cdot (2^4 -1) = 15M$ 
using a constant small search area of $\pm M$ pixels. Please note that
in some images you may know that the disparity always will be negative
while in other images it may (due to cropping) be both positive and
negative. 
\medskip

Several images show scene with depth discontinuities.  We recommend
that you do {\bf not} detect discontinuities explicitly. Partial visible (occluded) 
areas cling to discontinuities. They show connected groups of
unmatchable pixels. Advanced solutions may include an explicit
registering of such.  However, we recommend {\bf not} to do this
unless you already has a working system without.
\medskip

You are most welcome to compare your method with stereo algorithms
available on the internet or as OpenCV-implementation, but using such
implementations only will not be considered a solution to this
assignment. 
\medskip

To sum up, you should implement and test your own algorithm for
solving the stereo correspondence problem using normalised cross
correlation in a pyramid representation of the images. To enforce
smoothness, correspondences resulting in highly deviating disparities
should be avoided. The required amount of experiments and
documentation is described below.   





\section{Testing}
To validate and document your program you should use the images
{\em Tsubuka}, {\em Venus} and {\em Map} from the 2001-stereo data
set available at:
\url{http://vision.middlebury.edu/stereo/data/scenes2001/}. 
Two of these data comes with ground truth such that you may evaluate the
performance numerically. This should be done by for each image
computing: 
\begin{itemize}
\item The mean disparity error
\item The standard deviation of the disparity error
\item The number and fraction of large errors (error $\geq$ 4 pixels).
% \item The number and fraction of detected occluded pixels (if detected)
\end{itemize}

In addition you should describe your subjective evaluation of the
results including comments on where significant or strange errors
occur. Please note that I probably don't see the same as you - explain
to me what I should notice. Good observations and comments may show
more important than good results.  
\medskip

For development you are strongly encouraged to use a few (small and
simple synthetic) stereo images, but results on such data should not
be reported.  


\subsection{What to report}
Your report should include about two pages description of your
solution, the choices that you have made and a summation of your
results. 


For each of the image pairs with ground truth you should report the
numbers described above and your subjective evaluation. You should
also show a grey-coded image of the computed disparity surface.
Fotr the {\em Tsubuka} images, only your result and a subjective
evaluation should be given. 
\medskip  

In addition your code should be included (Please se first assignment
for the general information on submissions).
\end{document}

