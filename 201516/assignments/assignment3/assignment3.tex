\documentclass[a4paper,12pts]{article}

\usepackage[english]{babel}
\usepackage[latin1]{inputenc}
%\usepackage[dvips]{graphicx}
\usepackage{graphicx}
\usepackage{amsmath}
\usepackage{amssymb}
\usepackage{amsthm}
\usepackage{color}
\usepackage{hyperref}


\begin{document}

\title{Assignment 3: Content Based Image Retrieval \\
Vision and Image Processing}
\author{S�ren Olsen, Francois Lauze, and Hong Pan}
\date{December 1, 2014}
\maketitle


\noindent 
This is the third mandatory assignment on the course Vision
and Image Processing. The goal for you is to get familiar with
the Bag Of Words (BoW) principle and its use in content based image
retrieval systems.
\bigskip

{\bf This assignment must be solved in groups}. We expect that you will
form small groups of 2 to 4 students that will work on this assignment.
You have to pass this and the other 3 mandatory assignments in
order to pass the course.   If you do not pass this assignment, but you
have made a SERIOUS attempt, you will get a second chance of
submitting a new solution. 
\bigskip

{\bf The deadline for this assignment is Monday 4/1, 2016 at 20:00}. 
You must submit your solution electronically via the Absalon home
page. For general information on relevant software, requirement to the form
of your solution including the maximal page limit, how to upload on
Absalon etc, please see the first assignment.


\section*{Bag of Visual Words Content Based Image Retrieval}
The goal of the assignment is to implement a prototypical CBIR
system. We recommend the use of the
\href{http://www.vision.caltech.edu/Image_Datasets/Caltech101}{CalTech
  101} image database.
% One alternative is
% \href{http://www.vis.uky.edu/~stewe/ukbench}{University of Kentucky
%  Benchmark Data Set} but it is huge and has only four images per
% categories.  
We recommend that you (for a start) select a subset of say 10-20
categories. When you have checked that everything works you may extend
to more categories.  For each category, the set of images should be
split in two: A training set and a test set. The test set must not include
images in the training set.
\medskip

You should extract visual words using SIFT descriptors 
(ignoring position, orientation and scale) or similar descriptors
extracted at interest points. To compute the descriptors, we recommend
to use VLFeat's \texttt{sift}, but other options are possible.


\section{Codebook Generation}
In order to generate a code book, select a set of training
images. Then  Extract SIFT features from the training images (ignore
position, orientation and scaling). The SIFT features should be
concatenated into a matrix, one descriptor per row. Then you should
run the $k$-means clustering algorithm on the subset of training
descriptors to extract good prototype (visual word) clusters. 
A reasonable $k$ should be between 500 and 3000, depending on the
complexity of your data and the size of the training set.  You should
experiment with $k$ (but beware that this can be rather
time-consuming). 
\medskip

Once clustering has been obtained, classify each training descriptor to
the closest cluster centers) and form the bag of words (BoW) for each
image in the image training set.  
\medskip

Note that Matlab has its own implementation of $k$-means, called
\texttt{kmeans}, and so has Python, in the module
\texttt{scipy.cluster.vq}. For C++, we recommend the SHARK 
implementation, with header file \texttt{shark/Algorithms/KMeans.h}.
The Shark machine learning library, which should work on Windows,
Linux and MacOSX, is available at: 
\url{http://image.diku.dk/ shark/sphinx_pages/build/html/index.html} 
You need to have CMake and the boost library installed before you install Shark.


\section{Indexing}
The next step consists in indexing the content of the ``database'' 
(the quotation mark indicate that we are dealing more with a collection
of data rather than a structured database). For each image both in the test 
set and in the training set you should:

\begin{itemize}
\item Extract the SIFT descriptors of the feature points in the image,
\item Project the descriptors onto the codebook, i.e., for each
  descriptor the corresponding cluster prototype should be found,
\item Construct the generated corresponding bag of words, i.e, word histogram.
\end{itemize}

The result should be saved in a table that would contain, per entry at least
the file name, the true category, if it belongs to the training- or
test set,  and the corresponding bag of words / word histogram.
You may want to set up a database, if you know how to do it, but this
is not a requirement as simple structures will be enough for this
assignment.


\section {Retrieving}
Finally, you should implement retrieving of images using some of the
similarity measures discussed in the course slides. This may include
one or two of:
\begin{itemize}
\item common words
\item tf-ifd similarity
\item Bhattacharyya distance or Kullback-Leibler divergence
\end{itemize}

Your report should show commented results from a few experiments.
These should be repeated using images from the test set only and for
training images only. The results from the two set of experiments�ts
should be compared.
\medskip

How often do you retrieve the correct image and how often does it
fail.  What/how many failures might be explained by some obvious 
circumstances.  It is very important that you report what you have
done, the statistics of your results, and that you comment on the
ability of your system to perform CBIR.


\end{document}
