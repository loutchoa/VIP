\documentclass[10pt]{beamer}

% \mode<presentation> 
% { \usetheme[nat,dogma]{Frederiksberg} }

\mode<presentation> 
{
  \usetheme{Diku}
  \beamertemplatenavigationsymbolsempty
  \setbeamercovered{invisible}
%  \setbeamercovered{transparent}
}




% \usepackage[danish]{babel}
\usepackage[latin1]{inputenc}
\usepackage{times}
\usepackage[T1]{fontenc}
\usepackage[english]{babel}
\usepackage{hyperref}
\usepackage{animate}
%\usepackage{multimedia}
\usepackage{francois-preamble}
\usepackage{multirow}

\usepackage{multirow}
%\usepackage{movie15}

\newcommand{\cc}{{c\!\!,}}
\newcommand{\degr}[1]{{{#1}^\circ}}


\title{Vision and Image Processing:\\ Correspondence analysis}

\author[SIO] % (optional, use only with lots of authors)
{S�ren Olsen}

\institute[DIKU] % (optional, but mostly needed)
{
  Department of Computer Science\\
  University of Copenhagen
}

\date[2014-2015 B2] % (optional, should be abbreviation of conference name)
% {Research Presentation, Diku 2006}


% Insert page numbers
\pagenumbering{arabic}
\setbeamertemplate{footline}{\hspace{5pt}\insertpagenumber\vspace{10pt}}


\definecolor{gold}{rgb}{0.95,0.83,0.0}
\definecolor{orange}{rgb}{0.95,0.7,0.0}
% \definecolor{backblue}{rgb}{0.93,0.94,0.99}
\definecolor{backblue}{rgb}{0.95,0.94,0.99}
\definecolor{darkgreen}{rgb}{0.0,0.30,0.0}

\setbeamercolor*{background canvas}{bg=backblue} 



\newcommand{\myemph}[1]{{\color{blue}{#1}}}
\newcommand{\intrg}[1]{\int_{{#1}=-\infty}^\infty}
\newcommand{\intRR}{\int_{-\infty}^\infty}

\AtBeginSection[]
{
  \begin{frame}<beamer>{Outline}
    \tableofcontents[currentsection,currentsubsection]
  \end{frame}
}

\begin{document}
\maketitle

% would be cool with more images showing applications


%-------------------------------------------------------------------
%   Start slides
%-------------------------------------------------------------------




%----------------------------------------------
\begin{frame}
  \frametitle{Topics for today's lecture}
  \begin{itemize}
  \item Stereo vision, Epipolar geometry, Fundamental matrix etc \\[3mm]
  \item Correspondence analysis \\[3mm]
  \item Feature based vesus dense solutions \\[3mm]
  \item Scale-space and Coarse-to-fine \\[3mm]
  \item Reconstsructions from sparse data
  \end{itemize}
\end{frame}



% ----------------------------------
% \section{Stereo Vision}


\begin{frame}
  \frametitle{Multiple View Correspondences}
  \begin{center}
    \includegraphics[width=0.9\textwidth]{FIGURE/stereo}
  \end{center}
  If we can recover $x'$ from $x$ we can recover depth.
\end{frame}



%---------------------------------------------
\begin{frame}
  \frametitle{Image Correspondence}
  \begin{center}
    \includegraphics[width=0.9\textwidth]{IMAGES/stereo2images}
  \end{center}
  How do we match points from image 1 to image 2: You should have seen
  some of it before,  but this is not the end of the story!
\end{frame}




%---------------------------------------------
\begin{frame}
  \frametitle{Epipolar Constraints}
  \begin{center}
    \includegraphics[width=0.9\textwidth]{FIGURE/epipolarconstraint}
  \end{center}
  \begin{itemize}
  \item  Corresponding point for $x$ must lie in corresponding line $l'$
  \item  Corresponding point for $x'$ must lie in corresponding line $l$
  \end{itemize}  
\end{frame}



%----------------------------------------------
\begin{frame}
  \frametitle{Epipolar Constraints}
  \begin{center}
    \includegraphics[width=0.9\textwidth]{FIGURE/epigeom}
  \end{center}
  \begin{itemize}
  \item  Line connecting $O$ and $O'$: \myemph{baseline}
  \item Plane through baseline and $x$ or $x'$: \myemph{Epipolar Plane}
  \item Epipoles: intersection of baseline and image planes:
    projection of the other camera center.
  \item Epipolar Lines - intersections of epipolar plane with image
    planes (always come in corresponding pairs)
  \end{itemize}  
\end{frame}


%----------------------------------------------
\begin{frame}
  \frametitle{Example: Converging cameras}
  \begin{center}
     \includegraphics[width=0.9\textwidth]{FIGURE/convergecam}
  \end{center}
\end{frame}



%----------------------------------------------
\begin{frame}
  \frametitle{Calibrated Case}
   \begin{center}
    \includegraphics[width=0.65\textwidth]{FIGURE/epicalibrated}
  \end{center}
  Camera parameters known for the two cameras: calibration matrices $K$ and $K'$\\
\end{frame}



%----------------------------------------------
\begin{frame}
   \frametitle{The essential matrix $E$}
  \begin{itemize}
  \item  Let $y$ and $y'$ be 3D coordinates of the same scene point in
    the two (different) 3D-{\bf camera} coordinate systems. The two
    systems are related by a rotation and a translation. \\[3mm]
    $$y' = R(y-\bt)$$
  \item We will later show that $y$ and $y'$ are related by a 
    $3 \times 3$ matrix $E$ built from $R$ and $t$  \\[3mm]
    $$
    y^T E y' = 0.
    $$
  \item $E$ is called the \myemph{essential matrix} (Longuet-Higgins
    1981).
  \end{itemize}
\end{frame}



%----------------------------------------------
\begin{frame}
   \frametitle{The fundamental matrix $F$}
  \begin{itemize}
    \item The image coordinates ($x$ and $x'$) are related to the camera
      coordinates ($y$ and $y'$) through the calibration matrices $K$ and
      $K'$. Thus: 
      $$
          0 = y^T E y' = (K^{-1} x)^T E (K'^{-1} x') = 
          x^T (K^{-T} E K'^{-1} ) x' = x^T F x'
      $$
      where $x$ and $x'$ are the homogene representation of the
      corresponding points in image coordinates, and where $F$ is the
      \myemph{fundamental matrix}. \\[3mm]
  \item Given a sufficent number of corresponding image points ($x$ and $x'$),
    the fundamental matrix $F$ can be estimated.
  \item Given the internal intrinsic parameters ($K$ and $K'$) the
    essential matrix $E$ may be computed from $F$.
  \item Given $E$, the position and orientation of camera 1 vs camera
    2 (i.e., $R$ and $\bt$) can be recovered. 
  \end{itemize}
\end{frame}
 




%----------------------------------------------
\begin{frame}
   \frametitle{Calibration and reconstruction}
   \begin{itemize}
     \item Given (sufficient) image point correspondences, the
       fundamental matrix $F$ may be estimated
       using linear algebra. \\[3mm]
     \item $F$ has only 7 degrees of freedom (not 9) because it is
       defined up to scaling only, and because $\det (F) = 0$. \\[3mm]
     \item Linear estimation of $F$   is easy, but not accurate. In
       practice a non-linear post-optimization is needed. \\[3mm]
      \item Given $F$, the stereo correspondence problem is reduced
        to a one-dimensional search along the epipolar lines. \\[3mm]
      \item Given projections $(x,y)$ of known 3D points $(X,Y,Z)$,
        the calibration matrix $K$ may be estimates using linear algebra. \\[3mm]
      \item Given $F$, $K$ and $K'$, reconstructions of 3D points is
        possible (using linear algebra) from image point correspondences.
   \end{itemize}
   % {\color{orange}{More about this later}}.
\end{frame}




%----------------------------------------------
\begin{frame}
\frametitle{Proof of $x_R^{\top}Ex_L = 0$}
\begin{center}
      \includegraphics[width=0.7\textwidth]{MyImages/epipolarGeom.jpg}
\end{center}
We have that the camera coordinate systems are related by:
\begin{displaymath}
   {\bf P}_R = R ( {\bf P}_L - {\bf T})
\end{displaymath}

\begin{definition}
  The coplanarity condition:  
  ${\bf P}_L$, ${\bf T}$, and 
  ${\bf P}_L - {\bf T}$ are all in the epipolar plane. Then, also
   $R^{\top}{\bf P}_R$  is within the plane.
\end{definition}

\end{frame}



%----------------------------------------------
\begin{frame}
\frametitle{The cross-product}
The cross-product between two vectors ${\bf a}$ and ${\bf b}$ is a
vector that is perpendicular to both:

\begin{displaymath}
   {\bf a} \times {\bf b} \;=\; 
   \left ( 
   \begin{array}{c}
   -a_3 b_2 + a_2 b_3 \\ a_3 b_1 - a_1 b_3 \\ -a_2 b_1 + a_1 b_2
    \end{array}
    \right )
    \;=\; S {\bf b}
\end{displaymath}
where \\[3mm]
\begin{displaymath}
   S \;=\; [a]_x \;=\; \left [
     \begin{array}{c c c}
        0    & - a_3 & a_2  \\ 
        a_3 &    0    & -a_1 \\
      -a_2 &  a_1   &  0
     \end{array}
   \right ]
\end{displaymath}

$\mbox{}$ \\[2mm]
We see that $S$ is an anti-symmetric and rank deficient matrix. $S$
has rank 2.
\end{frame}



%----------------------------------------------
\begin{frame}
 \frametitle{Proof cont. 2}
Because  ${\bf P}_L$, ${\bf T}$, and ${\bf P}_L - {\bf T}$ all are in
the epipolar plane we can write:\\[3mm]

\begin{eqnarray*}
 0 & = &   ({\bf P}_L - {\bf T}) ^{\top} {\bf T} \times {\bf P}_L \\[2mm]
    & = &  (R^{\top}{\bf P}_R) ^{\top}{\bf T} \times {\bf P}_L \\[2mm]
    & = &  (R^{\top}{\bf P}_R) ^{\top} S {\bf P}_L\\[2mm]
    & = & {\bf P}_R^{\top} R S {\bf P}_L \\[2mm]
    & = & {\bf P}_R^{\top} E {\bf P}_L \\[2mm]
\end{eqnarray*}
where we have used that ${\bf P}_R = R ( {\bf P}_L - {\bf T})$ and 
$E = RS$.  

Since $\mbox{rank}(S) = 2$, $\mbox{rank}(E) = 2$. 
\end{frame}



%----------------------------------------------
\begin{frame}
\frametitle{The fundamental matrix equation once more}
We have now established the Essential matrix equation 
${\bf P}_R^{\top} E {\bf P}_L = 0$. To get to the fundamental matrix
equation we remember  the relation between the camera- and the image
coordinate systems:\\[3mm]

\begin{displaymath}
     K \;=\;
     \begin{pmatrix}
       \alpha & s & u_0\\
       0 & \beta & v_0\\
       0 & 0 & 1
     \end{pmatrix}
\end{displaymath}
\vspace{2mm}

Using ${\bf p}_L = K_L {\bf P}_L$ and ${\bf p}_R = K_R {\bf P}_R$ 
and defining 
\begin{displaymath}
  F \;=\; K_R^{-\top} E K_L^{-1}
\end{displaymath}
we finally get:\\[3mm]
\begin{displaymath}
  {\bf p}_R^{\top} F {\bf p}_L \;=\; 0
\end{displaymath}
\end{frame}



%----------------------------------------------
\begin{frame}
  \frametitle{Non horizontal Scan lines}
  \begin{itemize}
  \item If calibration known, the essential matrix provides epipolar
    constraints. \\[4mm] 
  \item What when cameras are in general position and calibration
    is unknown? \\[4mm]
  \item Non calibrated views: Estimate the fundamental matrix. \\[4mm] 
  \item Knowing Essential or Fundamental matrix allows (almost) for
    image rectification. 
  % \item To know more on fundamental matrices: 
  % \underline{\href{http://danielwedge.com/fmatrix}{follow the link!}}
  \end{itemize}
\end{frame}


%----------------------------------------------
\begin{frame}
  \frametitle{Projective Rectification}
  \begin{columns}
    \column{0.6\textwidth}
    \includegraphics[width=\textwidth]{FIGURE/stereorect}
    \column{0.4\textwidth}
    \begin{itemize}
    \item Reproject onto a common plane parallel to line between camera centers
    \item Projections are homographies!
    \item Pixel motion is horizontal after reprojection.
    \item Cf Loop-Zhang, CVPR 1999 (Rectification is not easy)
    \end{itemize}
  \end{columns}
\end{frame}


%----------------------------------------------
\begin{frame}
  \frametitle{Projective Rectification example}
  \begin{center}
    \includegraphics[width=0.8\textwidth]{IMAGES/projrectexpl}
  \end{center}
\end{frame}



%----------------------------------------------
\begin{frame}
\frametitle{Correspondence analysis}
Problem statement:  
{\color{blue}{Establish pairs $({\bf p_L}, {\bf p_R})$ of image 
points ${\bf p_L}$ in the left image and ${\bf p_R}$ in the
right image such that both points are projections of the same physical
scene point}}.

  \begin{itemize}
  \item Correspondence analysis is the difficult part of stereo
    analysis \\[3mm]
  \item Correspondence analysis is the basic of many other
    applications, eg. stitching, geo-referencing, image
    alingnment/warping etc. \\[3mm] 
  \item Most mammalians have stereo vision \\[3mm]
  \item Except for auto-focus cameras, stereo is the most widely
    applied passive  technique for 3D-measurement.
  \end{itemize}
\end{frame}





%----------------------------------------------
\begin{frame}
  \frametitle{Problems}

  \begin{minipage}{0.4\textwidth}
  \begin{itemize}
  \item Intensity in corresponding points are not equal: $E_L \neq E_R$.
  \item Many geometric properties, eg. orientation, are not preserved
    under perspective projections. 
  \item Occlusions: Things/areas visible en one image is invisible in
    the other.
  \item Double nail illusion
  \end{itemize}
  \end{minipage}
  \begin{minipage}{0.5\textwidth}
  % Show image pair with large occlusions
  % \begin{center}
  \includegraphics[width=\textwidth]{FIGURE/probilust1.jpg}
  % \end{center}
  \end{minipage}

\end{frame}





%----------------------------------------------
\begin{frame}
  \frametitle{More problems}
 \begin{itemize}
  \item Lack of texture/structure/intensity variation makes 
    makes feature matching difficult and intensity comparison
    vulnerable. 
  \item Complexity of correspondence problem: Given point in one
    image, how many possible mathes exist in the other image ?
  \end{itemize}

  \begin{center}
  \includegraphics[width=\textwidth]{FIGURE/kompleksitet.jpg}
  \end{center}

\end{frame}





%----------------------------------------------
\begin{frame}
  \frametitle{Simplifying assumptions}
  \begin{itemize}
  \item Intensities are similar, eg. $|E_L - E_R| \leq \theta$ or are
    spartially correlated (more later). \\[3mm]
  \item Fundamental matrix is estimated $\implies$ matching is reduced
    to 1D along epipolar lines. \\[3mm]
  \item The world consist of solid textured surfaces. Thus, the
    disparity is a single-valued function, and there exist a {\em
      unique} solution to the correspondence problem. \\[3mm]
  \item Occlusions and depth discontinuities do not exist. \\[3mm]
  \item Ordering: Corresponding points appear in the same order along
    the epipolar lines. \\[3mm]
  \item The magnitude of the disparity gradient is limited (for humans
    to about 1).    
  \end{itemize}
\end{frame}





%----------------------------------------------
\begin{frame}
\frametitle{Example: Large disparity gradient 1}

% Large disparity gradient (0.6):
\begin{center}
  \includegraphics[width=0.6\textwidth]{IMAGES/largegradient.pdf}
\end{center}

Humans can fuse random dot sterograms with no structural information.
Image has a disparity gradient of 0.6. Humans cannot fuse images with
gradient larger than 1.
\end{frame}


%----------------------------------------------
\begin{frame}
\frametitle{Example: Large disparity gradient 2}

\begin{center}
  \includegraphics[width=0.9\textwidth]{IMAGES/StereoPair-of-Bidonville.jpg}
  \end{center}

\end{frame}





%----------------------------------------------
\begin{frame}
\frametitle{Example: What surface ?}

% stero image of tree - do we have surfaces ?
  \begin{center}
  \includegraphics[width=0.9\textwidth]{IMAGES/loreo01.jpg}
  \end{center}


\end{frame}





%----------------------------------------------
\begin{frame}
\frametitle{Complexity}
Assume $n$ points along both epipolar lines. $N(n)$ is number
of solutions.
  \begin{center}
    \includegraphics[width=0.4\textwidth]{IMAGES/kompleksitet.jpg}
  \end{center}
 
Without assumptions:  
$N(n) \;=\; 2^{n^2}$. 
$N(4) \;=\; 65536$ \\[2mm]

Assume that each point may match at most one other point:
$N(n) \;= \; \sum_{i = 0}^{n} \frac{(n!)^2}{(n-i)! (i!)^2}$. 
$N(4) \;=\; 204$. \\[2mm]

Assume ordering, ie. 
$x_L^1 \leq x_L^2 \; \Rightarrow \; x_R^1 \leq x_R^2$, and uniqueness:  \\
$N(n) \;=\; \frac{(2n)!}{(n!)^2}$. 
$N(4) = 70$. \\[2mm]

% Assume strong ordering: All L-points match exactly one R-point: \\
Assume all L-points match exactly one R-point: \\
$N(n) \;=\; n!$. 
$N(4) \;=\; 12$. \\[2mm]

Assume strong ordering and uniqueness: \\
$N(n) \;=\; 1$.

\end{frame}



% \section{Feature based vesus dense solutions}



%----------------------------------------------
\begin{frame}
  \frametitle{Correspondence analysis}
  \begin{itemize}
  \item {\color{red}{Dense intensity based methods}} may be accurate but is very
    noise sensitive and have a small capture area. Also, they may be
    computationally expensive. \\[3mm]
   \item {\color{red}{Sparse feature based methods}} is faster, more reliable, and
     have larger capture area, but results in scattered depth information.\\[3mm]
  \item {\color{blue}{Very local}} dense features as edge often has a short descriptor. \\[3mm]
  \item {\color{blue}{Less local}} features (as SIFT) is less dense, but often has a
            more expressive descriptor. \\[3mm]
  \item {\color{blue}{Large features}} (as image segments) are few and
    more easy to match. 
  \end{itemize}
\end{frame}



%----------------------------------------------
\begin{frame}
  \frametitle{Area based stereo}
  \begin{itemize}
  \item Pixelwise intensity comparison does not work. Areas, say 
    $7 \times 7$, or $13 \times 13$ are used. Larger windows implies
    better robustness, less precision and larger vulnerability to occlusions.
  \item All R-windows centered and displaced along the epipolar line
    are compared to the L-window centered at the point in question and
    the best is chosen.
  \item Typical measure: {\color{green}{Normalized cross-correlation}}
  \end{itemize}

\begin{center}
  \includegraphics[width=0.9\textwidth]{FIGURE/arealstereo.jpg}
\end{center}
\end{frame}


%----------------------------------------------
\begin{frame}
  \frametitle{Disparity Map By Dense Block Matching\footnote{Slide adapted from  Derek Hoiem}}
  \begin{center}
    \includegraphics[width=\textwidth]{IMAGES/ddmapsblockmatch}
  \end{center}
  \begin{itemize}
  \item Window size 3: Noisy but detailed.
  \item Window size 20: smoother, but missing details.
  \end{itemize}
\end{frame}



%----------------------------------------------
\begin{frame}
\frametitle{Cross-correlation}
The cross-correlation between two continuous functions (with limited
square integral) is defined by:
\begin{displaymath}
  h(x) = (f \circ g)(x) \;\;=\;\; \int
        f^{\star}(\alpha) g(x \;+\; \alpha) d\alpha
\end{displaymath}

Discrete normalized 2D cross correlation is deined by:
\begin{displaymath}
\frac{1}{n} \sum_{(x,y) \in \Omega} 
    \frac{(f(x,y) - \bar{f}) \cdot  (g(x + \alpha,y + \beta) - \bar{g})}
    {\sigma_f \sigma_g}
\end{displaymath}
where $n$ is the number of pixels in $\Omega$, $\bar{f}$ is thge mean
value of $f$ in $\Omega$, $\sigma_f$ is the standard deviation of $f$
within $\Omega$ (and similarly for $g$).
\medskip

Cross-correlation is used in {\color{red}{Template matching}}
where we are searching for positions in $f(x,y)$ where the
signal/image is identical/similar to the prototype $g(x,y)$.  
Such positions can be identified as the local maxima's of 
$(f \circ g)(x,y)$.
\end{frame}




%----------------------------------------------
% \begin{frame}
% Cross-correlation between an image and a prototype and the image 
% patches centered at the local maximum.  \\
%
% \begin{tabular}{l c r}
%   \includegraphics[width=4.8cm]{FIGURE/crowd.eps} 
%   & \hspace{2mm} & 
%%  \begin{center}
%       \includegraphics[width=0.35cm]{FIGURE/face.eps} \hspace{2.0cm}
%%  \end{center}
%   \\
%   \includegraphics[width=4.8cm]{FIGURE/correlres.eps} 
%   & \hspace{2mm} & 
%   \includegraphics[width=4.8cm]{FIGURE/corrresult.eps} 
% \end{tabular}
%
% \end{frame}





%----------------------------------------------
\begin{frame}
  \frametitle{Feature base stereo}
  \begin{itemize}
  \item Which feature ? \\[3mm]
  \item For depth reconstructions, the density of points should be
    maximized. Often edge points localized with sub-pixel accuracy is
    chosen. \\[3mm]
  \item For matching efficiency, the quality (disambiguation power) of
    the descriptor should be maximized.  This is computational more
    expensive. Often corners or blobs with SIFT/HOG-like descriptors
    are used. \\[3mm]
  \end{itemize}

In practice hybrid systems may be used:  Sparse descriptive feature
points are used for fundamental matrix estimation.  Edge points are
used to obtain an initial reconstruction. Finally, intensity based
methods are used for fine-tuning and filling-in.
\end{frame}



% \section{Scale-space and Coarse-to-fine}



%----------------------------------------------
\begin{frame}
 \frametitle{3-Camera stereo}

The use of 3 cameras in stereo vision, and assuming all fundamental
matrices known, makes the correspondence analysis more easy and
robust. 

\begin{center}
      \includegraphics[width=0.6\textwidth]{FIGURE/kamera3.jpg}
\end{center}

\end{frame}



%----------------------------------------------
\begin{frame}
\frametitle{Scale-Space - repetition}
Often we don't know the size of the things we are imaging, so we have
to use both large and small filters when we analyze images. In
practice we represent each image at a number of scales. \\[5mm]

Please check previous slides for the definition and details on
scale-space. \\[5mm] 

% At small scales we only blurr the images slightly to attenuate noise.
% At large scale we blurr hevily to ensure that only large scale
% structures survive. \\[3mm]
% 
% The 2D-Gaussian filter has a number of unique properties, that makes
% it the default smoothing filter.  \\[3mm]

To save space and (in particular) time we subsample the smoothed image
versions. The result is an image pyramid.

% Often the large scale images are subsampled (according to the sampling
% theorem) to save space and to improve processing time.  
\end{frame}




%----------------------------------------------
% \begin{frame}
% \frametitle{Pyramid representations}
% A typical pyramid representation may involve 3-10 pyramid
% levels obtained by successive Gaussian low-pas-filtering and 
% sub-sampling with a reduction factor of 2.
% 
% \begin{center}
%   \includegraphics[width=6cm]{FIGURE/pyramide.eps}
% \end{center}
%
% \end{frame}






%----------------------------------------------
% \begin{frame}
% \frametitle{How to do it in MATLAB}
% 
% A banale implementation with known number of pyramid levels: \\[3mm]
% 
% {\texttt
% \noindent
% I0 = imread(inputimage); \\
% I1 = impyramid(I0, 'reduce'); \\
% I2 = impyramid(I1, 'reduce'); \\
% I3 = impyramid(I2, 'reduce'); \\
% }
% \bigskip
%
% The code below does not work in MATLAB because Pyr is a cube and not a  
% pyramid: \\[3mm]
% 
% {\texttt
% \noindent
% gaussfilt = fspecial('gaussian', 5, 1.4); \\
% Pyr          = zeros(dimx, dimy, Nscales); \\
% Pyr(:,:,1)  = Image; \\
% for scale = 2:Nscales \\
% \hspace{1cm}  smo  = 
%      imfilter(Pyr(:,:,scale-1), gaussfilt, 'replicate', 'same'); \\
% \hspace{1cm}  Pyr(:,:,scale) = smo(1:2:end,1:2:end); \\
% end 
% }
% \end{frame}





%-------------------------------------------------------------------
% \begin{frame}
% \frametitle{The isotropic Gaussian filter} 
% \begin{displaymath}
%   G_{\sigma}(x,y) = \frac{1}{2\pi \sigma^2} 
%                    e^{- \frac{x^2 + y^2}{2 \sigma^2}}
% \end{displaymath}
% have a large number of unique features (in addition to being
% linear, position invariant, isotropic, separable) including:
% \begin{itemize}
%  \item $G_{\sigma_1} \star G_{\sigma_2} = 
%        G_{\sqrt{\sigma_1^2 + \sigma_2^2}}$ 
%  \item $G_{\sigma} \bowtie G_{\frac{1}{2 \pi\sigma}}$ 
%  \item $\lim_n F \star^n F \; = \; G_{\sigma}$
%  \item $G_{\sigma}$ is simultaneously optimally concentrated
%        in the spatial as well as the frequency domain.
%  \item $G_{\sigma}$ is the solution to the heat equation 
% %       (see proof in J\"{a}hne section 5.3.1):
%        $f_{t} = k \cdot [ f_{xx} + f_{yy} ]$, with time defined by
%        $t = \frac{\sigma^2}{2}$.
%  \item $G_{\sigma}$ is the only filter satisfying both the
%        {\em semi-group property} and the 
%        {\em minimum-maximums principle} 
%%     (see J\"{a}hne section 5.3.2).
%        Both features are essential for a multi-scale representation.
% \end{itemize}
% \end{frame}


%-------------------------------------------------------------------
% \begin{frame}
% \frametitle{Scale-space time series of a signal}
%
% \begin{center}
%   \includegraphics[width=9.0cm,height=7.0cm]{FIGURE/scspfig.eps}% 
% \end{center}
%
% Each signal is lifted with a value about 20 above the previous 
% scale-space slice to make the evolution of the signal more visible.
% \end{frame}





%----------------------------------------------
\begin{frame}
\frametitle{Coarse-to-fine}
Pyramid-based coarse to fine approaches:
 
\begin{itemize}
  \item reduce the time complexity from $\mathcal{O}(N)$ to
    $\mathcal{O}(\log(N))$.  \\[3mm]
  \item reduce the complexity of the correspondence problem with a
    large factor (see previous slide). \\[3mm]
  \item Facilitates global operations using only local computations \\[3mm]
\end{itemize}
\bigskip

Principle: Use approximate solutions obtained at higher pyramid level
to constrain the search at lower levels.
\end{frame}


%----------------------------------------------
\begin{frame}
\frametitle{Coarse-to-fine Stereo}

\begin{minipage}{0.30\textwidth}
In practice the disparity may be large (several hundred pixels) and
have large variation (eg. from -50 to +50 pixels).  To reduce the size
of the search area we need an estimate of the disparity. 
\end{minipage}
\begin{minipage}{0.69\textwidth}
\begin{center}
  \includegraphics[width=0.75\textwidth]{FIGURE/grovfin.jpg} 
\end{center}
\end{minipage}

{\footnotesize
\begin{center}
{\color{darkgreen}{Method: Succesive smoothing and downsampling}} 
\end{center}
The total space requirement is: 
\begin{displaymath}
  1 \;+\; \frac{1}{4} \;+\; \frac{1}{16} \;+\; \cdots \;<\; \frac{4}{3}
\end{displaymath}
}
\end{frame}



%----------------------------------------------
\begin{frame}
%  \frametitle{Example}
\begin{center}
  \includegraphics[width=0.8\textwidth]{MyImages/EdgePyramidStereo2.pdf} 
\end{center}
\end{frame}




%----------------------------------------------
\begin{frame}
  \frametitle{Surface interpolation}
%   \begin{itemize}
%   \item Region growing (from point values)
%   \item Planar surface interpolation using Delaunay triangulation
%   \item Thin plate spline approximation using iterative updating. 
%   \end{itemize}

It is possible to fit a thin plate spline surface to the estimated
sparse depth measurements. Usually, an (slow) iterative updating is
used. To speed up a simple and fast method is used to obtain a initial 
estimate. 
\bigskip

Surface interpolation is too advanced for this course.
\end{frame}




%----------------------------------------------
\begin{frame}
\frametitle{Reconstruction on few data}
The results below are produced by the MATLAB-program
{\color{blue}{interp(.)}} that you may use in assignment 3.
\begin{center}
  \includegraphics[width=7.0cm]{MyImages/RECfewdata.jpg} 
\end{center}
Very sparse data, like SIFT-points makes reconstruction hard.
\end{frame}



%----------------------------------------------
\begin{frame}
\frametitle{Reconstruction on more data}
Dense data like edge points makes reconstruction better.

\begin{center}
  \includegraphics[width=7.0cm]{MyImages/RECmanydata.jpg} 
\end{center}

\end{frame}



%----------------------------------------------
\begin{frame}
  \frametitle{Discontinuities and occlusion}
  \begin{itemize}
  \item It is possible to perform discontinuous regularization, where
    the smoothmess term is disregarded when the surface is bended more
    than some threshold. \\[3mm]
  \item Discontinuities are accompanied by occluded areas. If any
    point in an occluded area is matched it will be wrong. If not, the
    surface will be reconstructed more smooth than what it should. \\[3mm]
  \end{itemize}

\begin{center}
  \includegraphics[width=7.0cm]{MyImages/3d-occlusion.jpg} 
\end{center}
\end{frame}



%----------------------------------------------
\begin{frame}
\frametitle{Example}
{\tiny
The images in this and the next slides are from Scharstein, Szeliski:
{\em A taxonomy and Evaluation of Dense Two-Frame Stereo
  Correspondence Algorithms}, Int.Jour. of Comput.Vis. 47, 2002.
}

 \begin{center}
  \includegraphics[width=8.0cm]{MyImages/ImStolen.pdf} 
\end{center}
\end{frame}



%----------------------------------------------
\begin{frame}
% \frametitle{Example}
 \begin{center}
  \includegraphics[width=9.0cm]{MyImages/ImStolen2.pdf} 
\end{center}
\end{frame}


%----------------------------------------------
% \begin{frame}
%   \frametitle{Other methods}
% You may find many other approaches in the literature, eg.
% a Dynamic-programming approach, See Forsyth and Ponce 2ed. 
% Section 7.5.2.
%\bigskip
%
%  \begin{center}
%    \includegraphics[width=0.5\textwidth]{MyImages/keepitsimple.png}
%  \end{center}
% 
% For assignment 3, you are advised to do something simple that
% you might get to work rather than more advanced methods.
% \end{frame}




%----------------------------------------------
% \begin{frame}
%   \begin{center}
%    \includegraphics[width=\textwidth]{IMAGES/tired}
%  \end{center}
% \end{frame}



% Below is for the exercises
%----------------------------------------------
\begin{frame}
\frametitle{Pentagon}
   \begin{center}
    \includegraphics[width=0.45\textwidth]{MyImages/PentagonR.jpg}
    \hspace{2mm}
    \includegraphics[width=0.45\textwidth]{MyImages/PentagonL.jpg}
  \end{center}
\end{frame}

%----------------------------------------------
\begin{frame}
Output of 110-lines of code programs to verify assignment:
   \begin{center}
    \includegraphics[width=0.4\textwidth]{MyImages/PentagonR.jpg}
    \hspace{2mm}\vspace{3mm}
    \includegraphics[width=0.5\textwidth]{MyImages/showdispPentagon.jpg}
  \end{center}
\end{frame}

%----------------------------------------------
% \begin{frame}
% \frametitle{Venus}
%    \begin{center}
%    \includegraphics[width=0.45\textwidth]{MyImages/VenusR.png}
%    \hspace{2mm}
%    \includegraphics[width=0.45\textwidth]{MyImages/VenusL.png}
%  \end{center}
% \end{frame}

%----------------------------------------------
% \begin{frame}
% If you get something similar your can be satisfied.
%    \begin{center}
%     \includegraphics[width=0.4\textwidth]{MyImages/GtVenus.jpg}
%     \hspace{2mm}
%    \includegraphics[width=0.5\textwidth]{MyImages/showdispVenus.jpg}
%  \end{center}
% \end{frame}



%----------------------------------------------
\begin{frame}
   \begin{center}
    \includegraphics[width=\textwidth]{IMAGES/tired}
  \end{center}
\end{frame}


\end{document}

