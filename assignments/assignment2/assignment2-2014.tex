\documentclass[a4paper,12pts]{article}

\usepackage[english]{babel}
\usepackage[latin1]{inputenc}
%\usepackage[dvips]{graphicx}
\usepackage{graphicx}
\usepackage{amsmath}
\usepackage{amssymb}
\usepackage{amsthm}
\usepackage{color}
\usepackage{hyperref}


\begin{document}

\title{Assignment 2: Content Based Image Retrieval \\
Vision and Image Processing}
\author{Kim Steenstrup Pedersen, S�ren Olsen, and Jan Kremer}
\date{December 1, 2014}
\maketitle


\noindent This is the second mandatory assignment on the course Vision
and Image Processing. The goal is for you to get familiar with
programming for computer vision with a focus on image descriptors and
their use for building image query systems.

You have to pass this and the following mandatory assignments in order
to pass this course. There are in total 4 mandatory pass/fail
assignments. This is a \textbf{group assignment}, i.e., we expect that you will
form small groups of 2 to 4 students that will work on this assignment.


The deadline for this assignment is Monday December 15th 2014. You must submit
your solution electronically via the Absalon home page. Go to the
assignments list and choose this assignment and upload your solution
prior to the deadline. Remember to include your name and KU user name
(login for {KU}net) in the solution. If you do not pass the
assignment, having made a SERIOUS attempt, you will get a second
chance of submitting a new solution.

A solution consist of:
\begin{itemize}
\item Your solution source code (Matlab / Python scripts / C / C++
  code) with comments about the major steps involved in each Question
  (see below).
\item Your code should be structured such that there is one main file
  that we can run to reproduce all the results presented in your
  report. This main file can, if you like, call other files with
  functions / classes.
\item Your code should also include a README text file describing how
  to compile (if relevant) and run your program, as well as list of
  all relevant libraries needed for compiling or using your code. If
  we cannot make your code run we will consider your submission
  incomplete and you may be asked to resubmit.
\item The code, auxiliary files and README file should be put into a
  compressed archive file in either the ZIP or tar format (RAR is not
  allowed - we simply cannot read your archive).
\item A PDF file with notes detailing your answers to the questions,
  which may include graphs and tables if needed ({\bf Max 5 pages}
  text including figures and tables). Do NOT include your source code
  in this PDF file.
\end{itemize}



\subsection*{A note on relevant software}
We recommend that you select either Matlab / Python / C / C++ as the programming language
you use for your solutions for these assignments. We also recommend that you select the
language you are most confident in - the focus should be on learning the methods and not
learning a new programming language.

If you wish to use Matlab, the University of Copenhagen has a license agreement with
MathWorks that allow students to download and install a copy of Matlab on personal
computers. On the KUnet web site you find a menu item called Software Library
(Softwarebiblioteket) \href{https://intranet.ku.dk/selvbetjening/Sider/Software-bibliotek.aspx}{https://intranet.ku.dk/
\\ selvbetjening/Sider/Software-bibliotek.aspx}. Under
this menu item you can find a link to The Mathworks - Matlab \& Simulink +
toolboxes. Click this link and follow the instructions for how to install on your own
computer.

If you use Python or C / C++ we recommend that you use the OpenCV library
\url{http://opencv.org} and / or the VLFeat library
\url{http://www.vlfeat.org/}. Both libraries provide an extensive collection of
implementations of central computer vision algorithms. OpenCV provides an interface for
Python. Follow the installation instructions on these websites to install on your own
computer. OpenCV is also available via MacPorts on Mac\-OSX.

If you wish to program your solutions in C++ we recommend the Shark machine learning
library \href{http://image.diku.dk/shark/sphinx_pages/build/html/index.html}{http://image.diku.dk/shark/sphinx\_pages/build/
\\ html/index.html} which should
work on Windows, Linux and MacOSX. You need to have CMake and the boost library installed
before you install Shark.

\section*{Bag of Visual Words Content Based Image Retrieval}

The goal of the assignment is to implement a prototypical CBIR
system. We recommend the use of the
\href{http://www.vision.caltech.edu/Image_Datasets/Caltech101}{CalTech
  101} image database, although you might consider others such as the
\href{http://www.vis.uky.edu/~stewe/ukbench}{University of Kentucky
  Benchmark Data Set} (but it is huge and has only four images per
categories).  The visual words will be built from SIFT descriptors
(ignoring position, orientation and scale). To compute the
descriptors, we recommend to use VLFeat's \texttt{sift}.

\section{Codebook Generation}

In order to generate the code book, a training set of images must be
selected. Extract SIFT features from the training set (without
position, orientation and scaling). They should be concatenated into a
matrix, one descriptor per row. Then you will run the $k$-means
clustering algorithm on it (or a subset of these descriptors).

A reasonable $k$ should be between 500 and 3000, depending on the
complexity of your data and the size of the training set.  You should
experiment with $k$ (but beware that this can be rather
time-consuming).

Once clustering has been performed, perform vector quantization (by
projecting on the cluster centers) and form the bag of words for each
image in the image training set. 

Note that Matlab has its own implementation of $k$-means, called \texttt{kmeans}, and so has Python,
in the module \texttt{scipy.cluster.vq} and for C++, we recommend the SHARK
implementation, with header file \texttt{shark/Algorithms/KMeans.h}.


\section{Indexing}

The next step consists in indexing the content of the ``database'' ( I
put quotation mark as we are more dealing with a collection of data
rather than a structured database). The task consist in computing for
each image of our collection (and not just the training collection)
\begin{itemize}
\item its SIFT descriptors,
\item then project them to the codebook, i.e., find for each
  descriptor the corresponding cluster and cluster center,
\item generate the corresponding bag of words, i.e, word histogram.
\end{itemize}

This should be saved in a table that would contain, per entry at least
the file name and the corresponding bag of words / word histogram.
You may want to set up a database, if you know how to do it, but this
is not a requirement as simple structures will be enough for this
assignment.


\section {Retrieving}

Implement retrieving via some of the similarity measures discussed in
the course (not necessarily all):
\begin{itemize}
\item common words
\item tf-ifd similarity
\item Bhattacharyya distance and Kullback-Leibler divergence
\end{itemize}

Report results from a few experiments , successes and failures, and
comment.





\end{document}
