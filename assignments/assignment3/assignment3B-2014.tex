\documentclass[a4paper,12pts]{article}

\usepackage[english]{babel}
\usepackage[latin1]{inputenc}
%\usepackage[dvips]{graphicx}
\usepackage{graphicx}
\usepackage{amsmath}
\usepackage{amssymb}
\usepackage{amsthm}
\usepackage{color}
\usepackage{hyperref}


\begin{document}

\title{Assignment 3: Stereo correspondence analysis}
\author{Kim Steenstrup Pedersen, S�ren I. Olsen, and Jan Kremer}
\date{December 15}
\maketitle


\noindent This third mandatory assignment on the course Vision
and Image Processing is about implementation of a classical algorithm
for stereo correspondence analysis. \\[3mm]

You have to pass this and the following mandatory assignments in order
to pass this course. There are in total 4 mandatory pass/fail
assignments. This is a \textbf{group assignment}, i.e., we expect that you will
form small groups of 2 to 4 students that will work on this assignment.


The deadline for this assignment is Wednesday, January 7, 2015. You must submit
your solution electronically via the Absalon home page. Go to the
assignments list, choose this assignment and upload your solution
prior to the deadline. Remember to include your name and KU user name
(login for {KU}net) in the solution. If you do not pass the
assignment, having made a SERIOUS attempt, you will get a second
chance of submitting a new solution.

A solution consists of:
\begin{itemize}
\item Your solution source code (Matlab / Python scripts / C / C++
  code) with comments about the major steps involved in each question
  (see below).
\item Your code should be structured such that there is one main file
  that we can run to reproduce all the results presented in your
  report. This main file can, if you like, call other files with
  functions / classes.
\item Your code should also include a README text file describing how
  to compile (if relevant) and run your program, as well as a list of
  all relevant libraries needed for compiling or using your code. If
  we cannot make your code run, we will consider your submission
  incomplete and you may be asked to resubmit.
\item The code, auxiliary files and README file should be put into a
  compressed archive file in either the ZIP or tar format (RAR is not
  allowed - we simply cannot read your archive).
\item A PDF file with notes detailing your answers to the questions,
  which may include graphs and tables if needed ({\bf Max 5 pages}
  text including figures and tables). Do NOT include your source code
  in this PDF file.
\end{itemize}



\subsection*{A note on relevant software}
We recommend that you select either Matlab / Python / C / C++ as the programming language
you use for your solutions for these assignments. We also recommend that you select the
language you are most confident in - the focus should be on learning the methods and not
learning a new programming language.

If you wish to use Matlab, the University of Copenhagen has a license agreement with
MathWorks that allow students to download and install a copy of Matlab on personal
computers. On the KUnet web site you find a menu item called Software Library
(Softwarebiblioteket): \\\url{https://intranet.ku.dk/Sider/Software-bibliotek.aspx}. \\
Under
this menu item you can find a link to The Mathworks - Matlab \& Simulink +
toolboxes. Click this link and follow the instructions for how to install on your own
computer.

If you wish to program your solutions in C++ we recommend the use of
the CImg Library, which can be obtained at \url{http://cimg.sourceforge.net}



\section*{Stereo correspondence analysis}
The goal of this assignment is to implement a traditional edge
based algorithm for stereo correspondence analysis.  You may assume
that the stereo image pairs show richly textured piecewise smooth
surfaces with Lambertian reflectance properties and that the images are
rectified. 
\medskip

To maximize the number of features and allow a reconstruction of a
dense disparity image edge points should be used (SIFT-feature points
will be far too sparse). You may attribute any descriptor you like to
the extracted feature points. The simplest choice is just to use the
gradient orientation. You should try this before using more advanced
(higher dimensional) descriptors. 
\medskip


The feature point matching may be based solely on the descriptors
attributed to the feature points. You may find that descriptive
strength of the edge orientation descriptor is limited (and may not
always single out the correct matches).  If so, you are welcome to 
choose another descriptor or to use a more advanced matching
strategy, eg.: Require consistent two-way matching; accept only
candidate matches much better than the next best;  accept only
candidate matches with high support from neighboring matches etc.



\section{Coarse to fine and prior knowledge}
To handle numerically large disparities a coarse-to-fine approach
using  a (truncated) factor-2 pyramid representations of the images
should be applied. Here, each image in the pyramid is subsampled with
a factor of 2 (along each dimension) from the next larger image. The
disparity magnitude will be reduced with the same factor. Thus, the
disparity will be numerically small at the topmost pyramid level and 
a small search area (of $\pm M$) will suffice.
\medskip

Having estimated the disparity at the top pyramid level at the sparse
edge points we can reconstruct a dense disparity suface using
interpolation or approximation.  You are advised to apply the
Thin-Plate-Spline approximation routine {\em interp(.)} available at
the Absalon course home page. The program is in MATLAB, but I trust
that you may transform it into your favourite language.
\medskip

Next, you may upsample the reconstructed disparity image to the size of
the images at the pyramid level just below and double the disparity
values. This gives you a prior for the disparity at this pyramid level
and makes possible to keep the search constantly small ( $\pm M$).
This again will reduce the probability of false matches.
\medskip

Now repeating the above procedure iteratively you may finally obtain
matches at the ground pyramid level and reconstruct a dense disparity
surface at the original image resolution. In this assignment you are 
advised to keep the number of pyramid levels fixed to say 4. This will
make it possible to register disparity values as large as
$D_{\mbox{max}} \;=\; M \cdot (2^4 -1) = 15M$ using a constant small
search area of $\pm M$ pixels.



\section{Depth computation}
Since the images are rectified in some unknown manner, possibly
cropped, and since we know nothing about the calibration of the
cameras, we have no chance of  reconstructing Euclidean depth.  In
this assignment, you may visualize the resulting disparity values
using $k(D_0-d)$, where $k$ is a scaling factor (converting to
intensity values) and where $D_0$ is some translation constant making
the resulting values positive.  




\section{Testing}
To validate and document your program you should use:
\begin{itemize}
\item The {\em Pentagon} stereo image pair available at the Absalon
  course homepage.
\item As supplement use the synthetic {\em Venus}- stereo image pair
  also available on Absalon and at the Middelbury stereo database \\ 
  \url{http://vision.middlebury.edu/stereo/data/scenes2001/}
\end{itemize}

You are not requested to make a quantitative evaluation of your result
(although ground truth is available for the {\em Venus}-images). 
Instead  you should make a subjective qualitative evaluation.
For development you are welcome to supplement with a few (small and
simple) stereo images, but results on these should not be reported. 


\subsection{What to report}
For each image pair you should give numbers for the detected feature
points and the matched ones.  You should show the reconstructed
disparity surface for each image and you should give a short
qualitative evaluation of your result. Also please give a short
informal/subjective evaluation of your implementation. 
\medskip

It is important is that you explain what can be seen and noted in each image
you show. Don't trust that I see the same as you. Please also note
errors and successes.  Good observations and comments may show more
important than good results.  
\medskip

It is not mandatory to evaluate your algorithm on images with ground
truth (known exact disparity surface), but it will show your maturity
and scientific capabilities.  You may count how large a fraction of
matches that have a measured disparity more than 1 pixel from the true
value, you may show histograms of the error, as well as the mean and
standard deviation of the disparity error.  However, all this is
add-on, and should not be considered before you have a working program.
%\newpage



\subsection{Appendix A}
For those still not sure what to implement the pseudo-code below may be
a help. 

\begin{verbatim}
   L_1 = L;  R_1 = R;
   for i = 1..N
       EDL_i = detectEdgePoints(L_i);
       ORL_i = gradientOrientation(L_i);
       EDR_i = detectEdgePoints(R_i);
       ORR_i = gradientOrientation(R_i);
       if (i < N)
           L_(i+1) = downsample(L_i);
           R_(i+1) = downsample(R_i);
       end
    end
    M = SearchRadius;
    D_N = zeros(size(L_N);
    for i = N..1
        (points, values) = match(EDL_i, ORL_i, EDR_i, ORR_i, M);
        D_N = interp(D_i, points, values, maxitt, lambda);
        if (i > 1)
             D_(i-1) = upsample(D_i);
        end
    end
    output(D_1);
\end{verbatim}

\noindent
Since some stereo images are in color you may have to convert them to
gray, eg. using  $Y \;=\; 0.3*R + 0.59*G + 0.11*B$.  If you:

\begin{itemize}
   \item Don't understand this color transformation, 
   \item Don't know how to detect edge point or compute a gradient orientation,
   \item Don't know what up- and downsampling means or how it is done,
   \item Don't know why you need an anti-aliasing filter and how it is
     implemented using a convolution with a low-pass-filter,
   \item Are uncertain on how to code the matching
\end{itemize}
\begin{center}

{\bf \Large Then you are strongly advised to show up \\
at the exercises Wednesday December 17 at 15:15. }
\end{center}

\end{document}
