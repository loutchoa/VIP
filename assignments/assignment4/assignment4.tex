\documentclass[a4paper,12pts]{article}

\usepackage[english]{babel}
\usepackage[latin1]{inputenc}
%\usepackage[dvips]{graphicx}
\usepackage{graphicx}
\usepackage{amsmath}
\usepackage{amssymb}
\usepackage{amsthm}
\usepackage{color}
\usepackage{hyperref}


\begin{document}

\title{Assignment 4: Kanade-Lucas-Tomasi (KLT) Tracker. \\
Vision and Image Processing}
\author{Kim Steenstrup Pedersen, S�ren Olsen, and Jan Kremer}
\date{\today}
\maketitle


\noindent This is the fourth mandatory assignment on the course Vision
and Image Processing. The goal is for you to get familiar with
programming for computer vision with a focus on a classical 2 dimensional tracking algorithm.

You have to pass this and the following mandatory assignments in order
to pass this course. There are in total 4 mandatory pass/fail
assignments. This is a \textbf{group assignment}, i.e., we expect that you will
form small groups of 2 to 4 students that will work on this assignment.

The deadline for this assignment is Friday 23/01 2015. You must submit
your solution electronically via the Absalon home page. Go to the
assignments list and choose this assignment and upload your solution
prior to the deadline. Remember to include your name and KU user name
(login for {KU}net) in the solution. If you do not pass the
assignment, having made a SERIOUS attempt, you will get a second
chance of submitting a new solution.

A solution consist of:
\begin{itemize}
\item Your solution source code (Matlab / Python scripts / C / C++
  code) with comments about the major steps involved in each Question
  (see below).
\item Your code should be structured such that there is one main file
  that we can run to reproduce all the results presented in your
  report. This main file can, if you like, call other files with
  functions / classes.
\item Your code should also include a README text file describing how
  to compile (if relevant) and run your program, as well as a list of
  all relevant libraries needed for compiling or using your code. If
  we cannot make your code run we will consider your submission
  incomplete and you may be asked to resubmit.
\item The code, auxiliary files and README file should be put into a
  compressed archive file in either the ZIP or tar format (RAR is not
  allowed - we simply cannot read your archive).
\item A PDF file with notes detailing your answers to the questions,
  which may include graphs and tables if needed ({\bf Max 5 pages}
  text including figures and tables). Do NOT include your source code
  in this PDF file.
\end{itemize}



\subsection*{A note on relevant software}
We recommend that you select either Matlab / Python / C / C++ as the programming language
you use for your solutions for these assignments. We also recommend that you select the
language you are most confident in - the focus should be on learning the methods and not
learning a new programming language.

If you wish to use Matlab, the University of Copenhagen has a license agreement with
MathWorks that allow students to download and install a copy of Matlab on personal
computers. On the KUnet web site you find a menu item called Software Library
(Softwarebiblioteket):\\ \href{https://intranet.ku.dk/selvbetjening/Sider/Software-bibliotek.aspx}{https://intranet.ku.dk/selvbetjening/Sider/Software-bibliotek.aspx}. \\
Under
this menu item you can find a link to The Mathworks - Matlab \& Simulink +
toolboxes. Click this link and follow the instructions for how to install on your own
computer.

If you use Python or C / C++ we recommend that you use the OpenCV library
\url{http://opencv.org} and / or the VLFeat library
\url{http://www.vlfeat.org/}. Both libraries provide an extensive collection of
implementations of central computer vision algorithms. OpenCV provides an interface for
Python. Follow the installation instructions on these websites to install on your own
computer. OpenCV is also available via MacPorts on Mac\-OSX.

If you wish to program your solutions in C++ we recommend the use of
the CImg Library, which can be obtained at \url{http://cimg.sourceforge.net}




\section*{Kanade-Lucas-Tomasi (KLT) Tracker}

The goal of this assignment is to implement and experiment with the classical Kanade-Lucas-Tomasi (KLT) feature 
tracking algorithm as described in \cite{Tomasi91a}. A copy of the original paper is available for download in Absalon. 
Read the paper carefully and then make an implementation of the outlined algorithm. We suggest that you use the 
parameter choices outlined in the paper. Please indicate in the report what choices you make for the different parameters 
including the choice of weight window function $w(\mathbf{x})$.

Extra: A possible extension is to include the features introduced by Shi and Tomasi 
\cite{Shi94a} ({OpenCV} has an implementation, see \verb|GoodFeaturesToTrackDetector|).


\section{Testing}

Run your implementation on a couple of video sequences of your choice. You can for instance use the sequence you find 
at \url{http://www.cs.toronto.edu/~fleet/research/DataSoftware/dudek.tar.gz} or use OpenCV video capture capability to 
capture your own video sequence. Do the parameter choices from \cite{Tomasi91a} generalize to your video sequences? 

Try to tune the parameters such that your implementation provides good results on your video sequences. 

Illustrate your results for both experiments with a couple of frames with the matched features drawn on top of the video image.

\section{Extension (Optional)}
Try to come up with an approach to tracking an object by combining all features on the object of interest. That is, how 
would you form an object region (i.e. a bounding box or circle) from the feature points on the object?

Explain your approach and illustrate your implementation on one or more video sequences by showing selected frames.


\bibliography{assignment4} \bibliographystyle{plain}



\end{document}
